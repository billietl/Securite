\documentclass[oneside,10pt]{article}
\usepackage[latin1]{inputenc}
\usepackage[francais]{babel}
\usepackage[francais]{layout}
\usepackage[OT1]{fontenc}
\usepackage{listings}
\usepackage{cite}
\usepackage{textcomp}
\usepackage{graphicx}

% Reglages du document
\lstset{language=bash, frame=single, breaklines=true, basicstyle=\ttfamily, keywordstyle=\bfseries}
\setlength{\hoffset}{-18pt}        
\setlength{\oddsidemargin}{0pt} % Marge gauche sur pages impaires
\setlength{\evensidemargin}{9pt} % Marge gauche sur pages paires
\setlength{\marginparwidth}{54pt} % Largeur de note dans la marge
\setlength{\textwidth}{481pt} % Largeur de la zone de texte (17cm)
\setlength{\voffset}{-18pt} % Bon pour DOS
\setlength{\marginparsep}{7pt} % S�paration de la marge
\setlength{\topmargin}{0pt} % Pas de marge en haut
\setlength{\headheight}{13pt} % Haut de page
\setlength{\headsep}{10pt} % Entre le haut de page et le texte
\setlength{\footskip}{27pt} % Bas de page + s�paration
\setlength{\textheight}{708pt} % Hauteur de la zone de texte (25cm)

\begin{document}

% Page de couverture
\title{Mise en place d'une Web Application Firewall}
\author{Louis BILLIET \\ Florent DAVID}
\date{10 D�cembre 2013}
\maketitle

\section{La faille mise en place}
Pour illustrer le fonctionnement d'un Web Application Firewall, nous avons ouvert une faille SQL.
Nous ne nous sommes pas cass�s la t\^ete pour y arriver : nous avons repris votre site \`a auditer et comment\'e un appel \`a \verb+addslashes+ surle traitement du formulaire de connection.
Ainsi, si on connais un identifiant de connection, on peut s'identifier dans utiliser de mot de passe.
Si le login utilis\'e est \verb+billietl+, il suffit d'entrer dans le champ login \verb+billietl' or 'a'='a+ (ou quelque chose dans ce genre l\`a, il n'y a pas qu'une technique) et de laisser le champ mot de passe vide..

\section{Le WAF utilis\'e}
\'Etant donn\'e que l'application est motoris� sur un serveur apache, nous utilisons le module \verb+security2+.
Vu le nombre de blogs sur internet qui en parle, il se peut que ce soit un module tr\`es efficace.
Le seul probl\`eme que l'on a rencontr\'e est le fait qu'on utilise Archlinux.
Le module y est disponible uniquement dans les d\'ep\^ots \verb+AUR+ (maintenu par les utilisateurs) et pas \`a jour.
Nous avons donc du installer \`a la main \`a partir des sources.
Ce qui n'\'etait pas insurmontable\ldots

\section{Comment on a combl\'e la faille}
Pour combler la faille, on pouvais voir deux angles d'attaque :
\begin{itemize}
\item d\'etecter et bloquer les mots-cl\'es SQL (select, update, insert, delete, from, and, or, \ldots).
Si on taille bien la r\`egle mise en place pour v\'erifier les requ\^etes SQL sensibles, on peut prot\'eger l'application d'une grande majorit\'e de failles.
Par contre, on risque fortement de bloquer des faux positifs (si on doit prot\'eger un forum par exemple, l'utilisation de certain des mots cl\'es de SQL est tout \`a fait l\'egitime).
Cette solution n'est donc pas vraiment envisageable dans la plupart des situations.
\item d\'etecter et patcher le caract\`ere \verb+'+.
Cet angle d'attaque est d\'ej\`a plus respectueuse de l'utilisateur l\'egitime qui aura le droit d'utiliser son apostrophe pr\'ef\'er\'ee sans probl\`emes.
On pourra aussi imaginer la m\^eme chose pour tous les meta-characters utilis\'e dans SQL.
Mais cela ne concerne pas l'exploitation que l'on illustre.
\end{itemize}


Cependant, par manque de comp\'etences en lecture de documentation, la r\`egle mise en place bloque simplement les requ\^etes contenant le caract\`ere \verb+'+ (dans sa version urlEncod�).
La r\`egle ressemble donc \`a : \\
\verb+SecRule REQUEST_BODY ``%27'' ``id:2,sanitiseArg:passwd,deny,log,msg:'SQL Injection Detected'''+

\section{Critiques}
L'utilisation d'un WAF permet de corriger des failles applicatives tr\`es vite.
Lorsqu'on la d\'etecte, il suffit d'ajouter une nouvelle r\`egle et de relancer le serveur pour combler la faille.
Ainsi, le site est prot\'eg\'e en attendant un correctif.


Cependant, maintenir une liste de r\`egle \'ecrite sur le volet peut devenir tr\`es vite fastidieux.
De plus, savoir \'ecrire des r\`egles qui n'impactera pas les utilisateurs l\'egitimes est tout une science qu'il est difficile de ma\^itriser en cas d'urgence.
L'utilisation syst\'ematique d'un WAF n'est donc pas, pour moi, une solution optimale sur le long terme, mais juste un outil permettant de palier \`a une urgence.

\end{document}
