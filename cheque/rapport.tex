\documentclass[oneside,12pt]{article}
\usepackage[latin1]{inputenc}
\usepackage[francais]{babel}
\usepackage[francais]{layout}
\usepackage[OT1]{fontenc}
\usepackage{fancybox}
\usepackage{ulem}
\usepackage{graphicx}
\usepackage{listings}
\usepackage{cite}
\usepackage{url}
\usepackage{textcomp}

% Réglages du document
\lstset{language=bash, frame=single, breaklines=true, basicstyle=\ttfamily, keywordstyle=\bfseries}

\begin{document}

% Page de couverture
\title{Proposition de solution : ch\`eque num\'erique}
\author{Louis BILLIET \\ Florent DAVID}
\date{25 Sept. 2013}
\maketitle

\section{Sujet}
Le but de l'exercice est d'imaginer un syst\`eme de ch\`eque num\'erique permettant \`a un client de payer un vendeur.
Le ch\`eque represente un ordre de virement de compte client au compte vendeur que la banque doit \'effectuer.
\\
\\
La connexion avec la banque est payante. Pour limiter les frais, nos vendeur souhaite grouper les ch\`eques. 
La connexion avec la banque est non garantie. (donc diff\'er\'e en cas d'echec).
\\
\\
Il existe divers canal de communication: entre vendeur et banque, entre client et vendeur.
\\
\\
L'outil de ch\`eque fournit par banque au client (Pas via  le vendeur).
\\
\\
But de l'exercice: Concevoir les 3 logiciels de mani\`ere securis\'e.
\\
\\
Attention:
\begin{itemize}
  \item Le client peut \^etre un bandit!
  \item Le vendeur peut \^etre un bandit!
\end{itemize}

\section{}

\section{Proposition de solution}

\section{Sceraii d'attaque par\'ees}
\subsection{Le client est malhonn\^ete}
D

\end{document}
