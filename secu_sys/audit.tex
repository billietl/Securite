\documentclass[oneside,10pt]{article}
\usepackage[latin1]{inputenc}
\usepackage[francais]{babel}
\usepackage[francais]{layout}
\usepackage[OT1]{fontenc}
\usepackage{listings}
\usepackage{cite}
\usepackage{textcomp}
\usepackage{graphicx}

% Reglages du document
\lstset{language=bash, frame=single, breaklines=true, basicstyle=\ttfamily, keywordstyle=\bfseries}
\setlength{\hoffset}{-18pt}        
\setlength{\oddsidemargin}{0pt} % Marge gauche sur pages impaires
\setlength{\evensidemargin}{9pt} % Marge gauche sur pages paires
\setlength{\marginparwidth}{54pt} % Largeur de note dans la marge
\setlength{\textwidth}{481pt} % Largeur de la zone de texte (17cm)
\setlength{\voffset}{-18pt} % Bon pour DOS
\setlength{\marginparsep}{7pt} % Séparation de la marge
\setlength{\topmargin}{0pt} % Pas de marge en haut
\setlength{\headheight}{13pt} % Haut de page
\setlength{\headsep}{10pt} % Entre le haut de page et le texte
\setlength{\footskip}{27pt} % Bas de page + séparation
\setlength{\textheight}{708pt} % Hauteur de la zone de texte (25cm)

\begin{document}

% Page de couverture
\title{Audit d'une passoire}
\author{Louis BILLIET \\ Florent DAVID}
\date{8 Janvier 2013}
\maketitle

\section{Comment cr\'eer un admin ?}
Lorsqu'on suis le lien \verb+http://localhost/?page=create_user+, et qu'on observe le code de la page, on peut voir un champ cach\'e ``status'' dont la valeur est \`a 1.
Cette valeur peut \^etre modifi\'e via le "DOM and style inpector" de firefox.
On voit dans le code source de la page \verb+http://localhost/?page=membres+ que 1 correspond \`a un visiteur, 2 \`a un \'etudiant et 3 \`a un enseignant.
On peut donc cr\'eer ais\'ement un nouvel utilisateur avec le status voulu afin d'en avoir les privil\`eges.

Que se passe-t-il si on forge une requ\^ete avec ce champ \'egal \`a 4? Ou bien \`a 0 ?

Un d\'etail nous frappe (aie!) lorsque l'on cr\'ee un utilisateur avec un status non ``out of bounds''. A la requ\`ete \verb+http://localhost/?page=membres&status=42+ 
\\l'interface affiche une \'erreur \verb+Notice: Undefined offset: 42 in /srv/http/modules/membres.php on line 30+.
Serait-ce une piste vers une autre faille?

\section{G\'en\'eration d'images pour ``prot\'eger'' l'adresse mail des gens ?}
Lorsqu'on suis le lien \verb+http://localhost/?page=membres+, les adresses mails sont des images... vers un script php, avec un paramettre pass\'e dans l'URL.
Chez moi, l'image g\'en\'er\'ee est blanche, quoi que je fasse. Bug ?
Toujours est-il que l'adresse e-mail est facilement lisible dans le code source de la page, ce qui n'emp\^echera pas un bot bien programm\'e de les r\'ecolter.

\section{Exploration !}
Toutes les images statiques sont stock\'ees dans le r\'epertoire \verb+static+.
Si on le consulte via le navigateur, on peut voir plein de fichiers dont\ldots ``login.html'' ???

M\^eme remarque pour \verb+http://192.168.12.94/modules/+, on y retrouve l'ensemble des fichiers php qui doivent \^etre utilis\'es sur le site.

Le meilleur, c'est qu'on puisse y naviger sans encombreset donc voir tous les rouages du site.

\section{XSS en force !}
Gr\^ace au formulaire d'inscription, on peut injecter une balise \verb+script+ avec du code.
Et ce, sans aucun encombres.
Suffit de la glisser dans le champ nom.
Si on r\'eussis \`a faire un \verb+alert("test");+, on peut tr\`es bien, par exemple, r\'ecup\'erer les sessions des autres utilisateurs\ldots

\section{Fichier suspicieux}
On a trouv\'e \verb+db.php+, que contient-il ?
\`A quoi sert-il ?
Serait-ce un script contenant les informations de connection ?

\end{document}
