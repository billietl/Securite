\documentclass[oneside,1�pt]{article}
\usepackage[latin1]{inputenc}
\usepackage[francais]{babel}
\usepackage[francais]{layout}
\usepackage[OT1]{fontenc}
\usepackage{listings}
\usepackage{cite}
\usepackage{textcomp}

% Réglages du document
\lstset{language=bash, frame=single, breaklines=true, basicstyle=\ttfamily, keywordstyle=\bfseries}

\begin{document}

% Page de couverture
\title{Proposition de solution : ch\`eque num\'erique}
\author{Louis BILLIET \\ Florent DAVID}
\date{25 Sept. 2013}
\maketitle

\section{Sujet}
Le but de l'exercice est d'imaginer un syst\`eme de ch\`eque num\'erique permettant \`a un client de payer un vendeur.
Le ch\`eque represente un ordre de virement de compte client au compte vendeur que la banque doit \'effectuer.
\\
\\
La connexion avec la banque est payante. Pour limiter les frais, nos vendeur souhaite grouper les ch\`eques. 
La connexion avec la banque est non garantie. (donc diff\'er\'e en cas d'echec).
\\
\\
Il existe divers canal de communication: entre vendeur et banque, entre client et vendeur.
\\
\\
L'outil de ch\`eque fournit par banque au client (Pas via  le vendeur).
\\
\\
But de l'exercice: Concevoir les 3 logiciels de mani\`ere securis\'e.
\\
\\
Attention:
\begin{itemize}
  \item Le client peut \^etre un bandit!
  \item Le vendeur peut \^etre un bandit!
\end{itemize}

\section{Conventions de nommage}
Soient :
\begin{itemize}
\item A.Kpub la cl� publique d'Alexandre.
\item B.Kpriv la cl� priv�e de Benjamin.
\item C.rib la relev� d'identit� banquaire de Charlie.
\item D.facture une facture emise pas Dominique.
\item E.paiement l'ordre de paiement venant d'Edouard.
\item S(F.rib, G.Kpub) le rib de Fr\'ederic sign� par Gilles.
\item S(H.Kpub) la signature du message courant avec la cl� publique de Henri.
\item C(I.rib, J.Kpriv) le rib de Ivan chiffr� avec la cl� priv�e de John.
\end{itemize}
Nous parlerons dans cet exercice de Victor, le vendeur, de Charles, le client, de Boris, le bandit et de Banque, la banque (nous ne ferons pas de publicit\'e ici).

\section{Analyse et pr\'esuppositions}
\'Etant donn\'e que le client et le vendeur payent ici avec un compte en banque, nous supposons qu'ils y ont au pr\'ealable ouvert un compte. Nous supposons ici que :
\begin{itemize}
\item Victor, Charles, Boris et Banque ont d\'ej\`a g\'en\'er\'e leurs paires de cl\'e publique/priv\'ee.
\item Banque fournis \`a chaque personne ouvrant un compte (qu'on apellera Pierre) un certain nombre d'informations, dont S(P.rib+P.Kpub, Bq.Kpriv).
\end{itemize}
Nous supposons de plus que Banque connais les informations suivantes :
\begin{itemize}
\item Bq.Kpub
\item Bq.kpriv
\item V.rib
\item V.Kpub
\item C.rib
\item C.Kpub
\end{itemize}
Nous supposons de plus que Victor (ainsi que Victor et Boris) connaissent les informations suivantes :
\begin{itemize}
\item Bq.Kpub
\item V.Kpub
\item V.Kpriv
\item V.rib
\item S(V.rib+V.Kpub, Bq.Kpriv)
\end{itemize}

\section{Proposition de solution}
Dans un premier temps, lorsque Victor annonce le montant \`a Charles, il lui transmet 3 informations :
\begin{itemize}
\item V.rib
\item le montant
\item un num\'ero de transaction unique pour le vendeur
\end{itemize}
Le message de Victor vers Charles aura donc la forme :\\V.rib+montant+transactionID


Pour payer Victor, Charles doit lui transmettre :
\begin{itemize}
\item V.rib
\item le montant
\item le num\'ero de transaction
\item C(C.rib+c.Kpub, Bq.Kpriv)
\end{itemize}
Donc, le ch\`eque aura la forme : \\C(vd.RIB+montant+transactionID, clt.Kpriv)+C(clt.RIB+clt.Kpub, bq.Kpriv)

\section{Sceraii d'attaque par\'es}

\end{document}
