\documentclass[oneside,10pt]{article}
\usepackage[latin1]{inputenc}
\usepackage[francais]{babel}
\usepackage[francais]{layout}
\usepackage[OT1]{fontenc}
\usepackage{listings}
\usepackage{cite}
\usepackage{textcomp}
\usepackage{graphicx}

% Reglages du document
\lstset{language=bash, frame=single, breaklines=true, basicstyle=\ttfamily, keywordstyle=\bfseries}
\setlength{\hoffset}{-18pt}        
\setlength{\oddsidemargin}{0pt} % Marge gauche sur pages impaires
\setlength{\evensidemargin}{9pt} % Marge gauche sur pages paires
\setlength{\marginparwidth}{54pt} % Largeur de note dans la marge
\setlength{\textwidth}{481pt} % Largeur de la zone de texte (17cm)
\setlength{\voffset}{-18pt} % Bon pour DOS
\setlength{\marginparsep}{7pt} % Séparation de la marge
\setlength{\topmargin}{0pt} % Pas de marge en haut
\setlength{\headheight}{13pt} % Haut de page
\setlength{\headsep}{10pt} % Entre le haut de page et le texte
\setlength{\footskip}{27pt} % Bas de page + séparation
\setlength{\textheight}{708pt} % Hauteur de la zone de texte (25cm)

\begin{document}

% Page de couverture
\title{Tracking des ``anonymes''}
\author{Louis BILLIET \\ Florent DAVID}
\date{8 Janvier 2013}
\maketitle

\section{Strat\'egie de tra\c cage}
Notre page de tracking consiste uniquement en une page php.
Selon la requ\^ete, on traite la (d\'e)connexion o\`u on affiche juste le nom de la personne que l'on a reconnu.
Nous reconnaissons les personnes selon 6 crit\`eres :
\begin{itemize}
\item cookies
\item user\_agent
\item adresse du client
\item adresse du proxy (s'il y en a)
\item adresse proxifi\'ee (s'il y en a)
\item contr\^ole de cache
\end{itemize}
Mis \`a part les cookies, nous enregistrons les diff\'erentes informations en base SQL avec le pseudo renseign\'e lors de l'enregistrement.
Lors de la d\'econnexion, nous cherchons en base le pseudo qui correspond le mieux au profil de la personne qui visite.

\section{Strat\'egie d'anonymisation}
Pour nous anonymiser, nous controns principalement les points qu'on exploite :
\begin{itemize}
\item Concernant les cookies : nous les supprimons apr\`es d\'econnexion.
\item Concernant le user-agent : nous utilisons \verb+modify headers+, un plugin pour firefox qui modifie \`a la vol\'ee les headers envoy\'es par firefox.
N'est-il pas classe d'utiliser un minitel overclock\'e pour naviger sur internet ?
\item Concernant l'adresse ip : on passe par un web-proxy gratuit (http://www.free\_proxy.fr par exemple).
Plus efficace aurait \'et\'e l'utilisation de TOR, mais c'est d'un autre niveau.
\end{itemize}
Pour aller plus loin, on aurait pu utiliser \verb+no-script+, un plugin pour firefox qui bloque les scripts selon une whitelist.
Mais il parait qu'on avait pas le droit\ldots

\end{document}
