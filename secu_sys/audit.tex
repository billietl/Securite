\documentclass[oneside,10pt]{article}
\usepackage[latin1]{inputenc}
\usepackage[francais]{babel}
\usepackage[francais]{layout}
\usepackage[OT1]{fontenc}
\usepackage{listings}
\usepackage{cite}
\usepackage{textcomp}
\usepackage{graphicx}

% Reglages du document
\lstset{language=bash, frame=single, breaklines=true, basicstyle=\ttfamily, keywordstyle=\bfseries}
\setlength{\hoffset}{-18pt}        
\setlength{\oddsidemargin}{0pt} % Marge gauche sur pages impaires
\setlength{\evensidemargin}{9pt} % Marge gauche sur pages paires
\setlength{\marginparwidth}{54pt} % Largeur de note dans la marge
\setlength{\textwidth}{481pt} % Largeur de la zone de texte (17cm)
\setlength{\voffset}{-18pt} % Bon pour DOS
\setlength{\marginparsep}{7pt} % Séparation de la marge
\setlength{\topmargin}{0pt} % Pas de marge en haut
\setlength{\headheight}{13pt} % Haut de page
\setlength{\headsep}{10pt} % Entre le haut de page et le texte
\setlength{\footskip}{27pt} % Bas de page + séparation
\setlength{\textheight}{708pt} % Hauteur de la zone de texte (25cm)

\begin{document}

% Page de couverture
\title{Audit d'une passoire}
\author{Louis BILLIET \\ Florent DAVID}
\date{8 Janvier 2013}
\maketitle

\section{Cr\'eation de comptes privil\'eg\'es}
\subsection{O\`u est la faille}
La page de \verb+create_user+ permet de cr\'eer un utilisateur, bien entendu.
Par contre, les privil\`eges accord\'es au nouvel utilisateur sont sp\'ecifi\'es dans un champ cach\'e du formulaire (nomm\'e \verb+status+).
Le probl\`eme est que ce champ n'est pas v\'erifi\'ee co\'e serveur lors de la cr\'eation de l'utilisateur.

\subsection{Cons\'equences}
\'Etant donn\'e que le champ \verb+status+ n'est soumis \`a aucun contr\^ole, un utilisateur malicieux peut tr\`es bien forger une requ\^ete avec un champ \verb+status+ modifi\'e.
D'autant plus que pour y parvenir, il lui suffira de modifier le code source de la page (tous les navigateurs performants proposent ce genre d'option de base) puis de s'inscrire.
Enfin, pour savoir quelle valeur utiliser, il lui suffira de regarder en d\'etails la page \verb+membres+.
Il verra tr\`es vite quelle valeur utiliser pour cr\'eer un nouveau compte enseignant.

\subsection{Recommandations pour colmater la faille}
Pour corriger ce probl\`eme, il faut que les droits accord\'es ne soient pas communiqu\'e avec les informations servant \`a ouvrir un nouveau compte.
Comprenez : n'utilisez pas de champs hidden pour \c ca.


Nous pr\'econisons une cr\'eation par d\'efaut de comptes sans privil\`eges et de cr\'eer une page d'administration pour pouvoir \'elever les droits d'un utilisateur.

\subsection{Sympt\^omes suppl\'ementaires}
Sur la page \verb+membres+, il est possible de faire un tri selon les droits des membres (invit\'es, \'etudiants, enseignants).
Ce tri se fait via un param\`etre pass\'e dans l'adresse de la page.
Il n'y a donc aucune difficult\'e \`a la changer pour une valeur qui ne correspond \`a rien.
Si on utilise la valeur 42, par exemple, la page nous affiche \verb+Notice: Undefined offset: 42 in /srv/http/modules/membres.php on line 30+.
Cette erreur de programmation permet une fuite d'informations qui fait tache dans l'interface.

\section{Protection anti-bot tr\`es peu efficace}
\subsection{O\`u est la faille}
\subsection{Cons\'equences}
\subsection{Recommandations pour colmater la faille}
Lorsqu'on suis le lien \verb+http://localhost/?page=membres+, les adresses mails sont des images... vers un script php, avec un paramettre pass\'e dans l'URL.
Chez moi, l'image g\'en\'er\'ee est blanche, quoi que je fasse. Bug ?
Toujours est-il que l'adresse e-mail est facilement lisible dans le code source de la page, ce qui n'emp\^echera pas un bot bien programm\'e de les r\'ecolter.

\section{Fuites d'informations}
\subsection{O\`u est la faille}
\subsection{Cons\'equences}
\subsection{Recommandations pour colmater la faille}
Toutes les images statiques sont stock\'ees dans le r\'epertoire \verb+static+.
Si on le consulte via le navigateur, on peut voir plein de fichiers dont\ldots ``login.html'' ???

M\^eme remarque pour \verb+http://192.168.12.94/modules/+, on y retrouve l'ensemble des fichiers php qui doivent \^etre utilis\'es sur le site.

Le meilleur, c'est qu'on puisse y naviger sans encombreset donc voir tous les rouages du site.

On a trouv\'e \verb+db.php+, que contient-il ?
\`A quoi sert-il ?
Serait-ce un script contenant les informations de connection ?

\section{Execution de javascripts provenants des utilisateurs}
\subsection{O\`u est la faille}
\subsection{Cons\'equences}
\subsection{Recommandations pour colmater la faille}
Gr\^ace au formulaire d'inscription, on peut injecter une balise \verb+script+ avec du code.
Et ce, sans aucun encombres.
Suffit de la glisser dans le champ nom.
Si on r\'eussis \`a faire un \verb+alert("test");+, on peut tr\`es bien, par exemple, r\'ecup\'erer les sessions des autres utilisateurs\ldots

\end{document}
