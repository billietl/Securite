\documentclass[oneside,10pt]{article}
\usepackage[latin1]{inputenc}
\usepackage[francais]{babel}
\usepackage[francais]{layout}
\usepackage[OT1]{fontenc}
\usepackage{listings}
\usepackage{cite}
\usepackage{textcomp}
\usepackage{graphicx}

% Reglages du document
\lstset{language=bash, frame=single, breaklines=true, basicstyle=\ttfamily, keywordstyle=\bfseries}
\setlength{\hoffset}{-18pt}        
\setlength{\oddsidemargin}{0pt} % Marge gauche sur pages impaires
\setlength{\evensidemargin}{9pt} % Marge gauche sur pages paires
\setlength{\marginparwidth}{54pt} % Largeur de note dans la marge
\setlength{\textwidth}{481pt} % Largeur de la zone de texte (17cm)
\setlength{\voffset}{-18pt} % Bon pour DOS
\setlength{\marginparsep}{7pt} % Séparation de la marge
\setlength{\topmargin}{0pt} % Pas de marge en haut
\setlength{\headheight}{13pt} % Haut de page
\setlength{\headsep}{10pt} % Entre le haut de page et le texte
\setlength{\footskip}{27pt} % Bas de page + séparation
\setlength{\textheight}{708pt} % Hauteur de la zone de texte (25cm)

\begin{document}

% Page de couverture
\title{Mise en place et contournement d'un IDS}
\author{Louis BILLIET}
\date{17 f\'evrier 2014}
\maketitle

\section{Mise en place}
\subsection{Service faible}
Le service attaqu\'e est un service web volontairement trou\'e de partout heberg\'e sur un LAMP \`a jour (le 16 f\'evrier 2014).

\subsection{IDS}
L'IDS est install\'e sur le m\^eme serveur qui heberge le service.
C'est moins s\'ecuris\'e mais pour les besoins du TP, \c ca suffira
\begin{itemize}
\item OS : Ubuntu 13.10 32 bits
\item IDS : Snort version 2.9.2.2
\item R\`egles : https://rules.emergingthreats.net/open/snort-2.9.2/rules/.
Mises \`a jour des r\`egles automatiquement via oinkmaster.
\end{itemize}

\section{Attaque envisag\'ee}
La page d'inscription est faible face aux attaques XSS.
Ins\'erer un script qui redirige automatiquement vers une autre page peut rendre indisponibles certaines pages.

\section{Ce que j'ai modifi\'e dans l'espoir de me faire d\'etecter}
\subsection{V\'erification des r\`egles}
Apr\`es avoir mis en place les r\`egles d'emergingthreats, aucune attaque n'\'etait d\'etect\'ee.
J'ai donc \'ecrit une r\`egle en m'inspirant de ce qui est \'ecrit sur cette page : http://www.symantec.com/connect/articles/detection-sql-injection-and-cross-site-scripting-attacks (section 3 ``Regular Expressions for Cross Site Scripting (CSS)'').
La r\`egle donne :
\begin{verbatim} alert tcp $EXTERNAL_NET any -> $HTTP_SERVERS $HTTP_PORTS (msg:"LOUIS XSS attaque (crade)";\end{verbatim}
\begin{verbatim} pcre:"/((\%3C)|<)[^\n]+((\%3E)|>)/I"; classtype:Web-application-attack; sid:9000; rev:1;) \end{verbatim}
Cette r\`egle ne levant toujours aucune alarme, j'ai donc \'ecrit une r\`egle qui l\`eve une alarme au moindre trafic sur le port 80 de mon serveur.
Ce qui donne :
\begin{verbatim}alert tcp any any -> $HTTP_SERVERS $HTTP_PORTS (msg:"activite sur le port 80"; sid:9000;rev:1;)\end{verbatim}
Vu que m\^eme dans cette situation, aucune alerte n'\'etait lev\'ee, j'ai remis en cause la configuration de snort.

\subsection{V\'erification de la configuration de snort}
J'ai v\'erifi\'e la plage d'adresse surveill\'ee, les ports surveill\'es, le profil du preprocesseur HTTP et la liste des r\`egles import\'ees.
Il n'y a rien, de mon point de vue, qui soit mal configur\'e.
Voici quand m\^eme ce que donne les parties concern\'ees par la conf :
\begin{verbatim}
ipvar HOME_NET 10.0.0.0/8
ipvar EXTERNAL_NET any
ipvar HTTP_SERVERS $HOME_NET

portvar HTTP_PORTS [80, ...] # ligne ecourtee

var RULE_PATH /etc/snort/rules

preprocessor http_inspect: \
    global iis_unicode_map unicode.map 1252 compress_depth 65535 decompress_depth 65535
preprocessor http_inspect_server: \
    server apache profile all ports { 80 }

include $RULE_PATH/community-web-php.rules
\end{verbatim}


\subsection{V\'erification de la d\'etection du trafic}
\'Etant donn\'e qu'aucune r\`egle ne veut lever d'alertes, je me suis donc dit que snort ne voyait pas le trafic.
Le fait que apache r\'eponde \`a mes requ\^etes prouve que le traffic est bien \'etablis.
De plus, wareshark et tcpdump montrent clairement l'\'echange qui s'op\`ere entre le navigateur de l'attaquant et apache\ldots

\section{Techniques d'\'evasion envisag\'ees}
\subsection{Modification de l'encodage}
En modifiant l'encodage des chevrons (utiliser \%3C au lieu de \textless, voire m\^eme repasser l'encodeur et envoyer \%253C), la regex s'occupant de d\'etecter la faille serait tromp\'ee.
\subsection{Fragmentation des paquets}
L'id\'ee est de modifier la taille du MTU de la carte r\'eseau (avec la commande ifconfig par exemple).
En jouant avec la MTU, on peut exp\'erer sectionner la requ\^ete au bon endroit pour que les deux chevrons se trouvent dans deux paquets TCP s\'epar\'es.
\subsection{Ajout d'espaces}
Afin de faciliter le jeu avec la MTU, on peut ajouter des espaces dans les balises pour forcer la d\'ecoupe des balises qu'on injecte.

\section{Bibliographie}
\begin{itemize}
\item Regex pour d\'etecter les attaques XSS :\\
http://www.symantec.com/connect/articles/detection-sql-injection-and-cross-site-scripting-attacks
\item Documentation de snort \`a propos des r\`egles :\\
http://manual.snort.org/node27.html
\item Documentation de snort \`a propos des pr\'eprocesseurs :\\
http://manual.snort.org/node17.html
\end{itemize}

\end{document}
