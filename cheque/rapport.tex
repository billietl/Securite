\documentclass[oneside,1�pt]{article}
\usepackage[latin1]{inputenc}
\usepackage[francais]{babel}
\usepackage[francais]{layout}
\usepackage[OT1]{fontenc}
\usepackage{listings}
\usepackage{cite}
\usepackage{textcomp}

% Réglages du document
\lstset{language=bash, frame=single, breaklines=true, basicstyle=\ttfamily, keywordstyle=\bfseries}

\begin{document}

% Page de couverture
\title{Proposition de solution : ch\`eque num\'erique}
\author{Louis BILLIET \\ Florent DAVID}
\date{25 Sept. 2013}
\maketitle

\section{Sujet}
Le but de l'exercice est d'imaginer un syst\`eme de ch\`eque num\'erique permettant \`a un client de payer un vendeur.


\section{Conventions de nommage}
Soient :
\begin{itemize}
\item A.Kpub la cl� publique d'Alexandre.
\item B.Kpriv la cl� priv�e de Benjamin.
\item C.rib la relev� d'identit� banquaire de Charlie.
\item D.facture une facture emise pas Dominique.
\item E.paiement l'ordre de paiement venant d'Edouard.
\item S(F.rib, G.Kpub) le rib de Fr\'ederic sign� par Gilles.
\item S(H.Kpub) la signature du message courant avec la cl� publique de Henri.
\item C(I.rib, J.Kpriv) le rib de Ivan chiffr� avec la cl� priv�e de John.
\end{itemize}

\section{Analyse et pr\'esuppositions}
\'Etant donn\'e que le client et le vendeur payent ici avec un compte en banque, on suppose qu'ils y ont au pr\'ealable ouvert un compte. Nous supposons ici que :
\begin{itemize}
\item Le client / le vendeur \`a d\'ej\`a g\'en\'er\'e sa paire de cl\'e publique/priv\'ee.
\item La banque fournis a chaque personne ouvrant un compte un certain nombre d'informations.
\end{itemize}



\section{Proposition de solution}

\section{Sceraii d'attaque par\'es}

\end{document}
