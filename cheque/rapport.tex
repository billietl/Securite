\documentclass[oneside,10pt]{article}
\usepackage[latin1]{inputenc}
\usepackage[francais]{babel}
\usepackage[francais]{layout}
\usepackage[OT1]{fontenc}
\usepackage{listings}
\usepackage{cite}
\usepackage{textcomp}

% Réglages du document
\lstset{language=bash, frame=single, breaklines=true, basicstyle=\ttfamily, keywordstyle=\bfseries}
\setlength{\hoffset}{-18pt}        
\setlength{\oddsidemargin}{0pt} % Marge gauche sur pages impaires
\setlength{\evensidemargin}{9pt} % Marge gauche sur pages paires
\setlength{\marginparwidth}{54pt} % Largeur de note dans la marge
\setlength{\textwidth}{481pt} % Largeur de la zone de texte (17cm)
\setlength{\voffset}{-18pt} % Bon pour DOS
\setlength{\marginparsep}{7pt} % S�paration de la marge
\setlength{\topmargin}{0pt} % Pas de marge en haut
\setlength{\headheight}{13pt} % Haut de page
\setlength{\headsep}{10pt} % Entre le haut de page et le texte
\setlength{\footskip}{27pt} % Bas de page + s�paration
\setlength{\textheight}{708pt} % Hauteur de la zone de texte (25cm)

\begin{document}

% Page de couverture
\title{Proposition de solution : ch\`eque num\'erique}
\author{Louis BILLIET \\ Florent DAVID}
\date{25 Sept. 2013}
\maketitle

\section{Sujet}
Le but de l'exercice est d'imaginer un syst\`eme de ch\`eque num\'erique permettant \`a un client de payer un vendeur.
Le ch\`eque represente un ordre de virement de compte client au compte vendeur que la banque doit effectuer.
\\
\\
La connexion avec la banque est payante. Pour limiter les frais, nos vendeurs souhaitent grouper les ch\`eques. 
La connexion avec la banque est non garantie (donc diff\'er\'e en cas d'echec).
\\
\\
Il existe divers canaux de communication: entre vendeur et banque et entre client et vendeur.
\\
\\
L'outil d'\'emission de ch\`eque est fournit par banque au client (pas via  le vendeur).
\\
\\
But de l'exercice: Concevoir les 3 logiciels de mani\`ere securis\'e : g\'en\'eration de ch\`eque, v\'erification du ch\`eque et encaissement du ch\`eque.
\\
\\
Attention:
\begin{itemize}
  \item Le client peut \^etre un bandit!
  \item Le vendeur peut \^etre un bandit!
  \item La banque ne peut \^etre qu'une banque! Merci la banque!
\end{itemize}

\section{Conventions de nommage}
Soient :
\begin{itemize}
\item A.Kpub la cl\'e publique d'Alexandre.
\item B.Kpriv la cl\'e priv\'ee de Benjamin.
\item C.rib la relev\'e d'identit\'e banquaire de Charlie.
\item D.facture une facture emise pas Dominique.
\item E.paiement l'ordre de paiement venant d'Edouard.
\item S(F.rib, G.Kpub) le rib de Fr\'ederic sign\'e par Gilles.
\item S(H.Kpub) la signature du message courant avec la cl\'e publique de Henri.
\item C(I.rib, J.Kpriv) le rib de Ivan chiffr\'e avec la cl\'e priv\'ee de John.
\item K.HRid l'identifiant ``Human Readable'' de Kevin.
\end{itemize}
Nous parlerons dans cet exercice de Victor, le vendeur, de Charles, le client, de Boris, le bandit et de Banque, la banque (nous ne ferons pas de publicit\'e ici).

\section{Analyse et pr\'esuppositions}
\'Etant donn\'e que le client et le vendeur payent ici avec un compte en banque, nous supposons qu'ils y ont au pr\'ealable ouvert un compte. Nous supposons ici que :
\begin{itemize}
\item Victor, Charles, Boris et Banque ont d\'ej\`a g\'en\'er\'e leurs paires de cl\'e publique/priv\'ee.
\item Banque fournis \`a chaque personne ouvrant un compte (qu'on apellera Pierre) un certain nombre d'informations, dont S(P.HRid+P.RIB+P.Kpub, Bq.Kpriv).
\end{itemize}
Nous supposons de plus que Banque connais les informations suivantes :
\begin{itemize}
\item Bq.Kpub
\item Bq.Kpriv
\item V.RIB
\item V.Kpub
\item V.HRid
\item C.RIB
\item C.Kpub
\item C.HRid
\end{itemize}
Nous supposons de plus que Victor (ainsi que Charles et Boris) connaissent les informations suivantes :
\begin{itemize}
\item Bq.Kpub
\item V.Kpub
\item V.Kpriv
\item V.RIB
\item V.HRid
\item S(V.HRid+V.rib+V.Kpub, Bq.Kpriv)
\end{itemize}

\section{Proposition de solution}
\subsection{format des communications}
Dans un premier temps, lorsque Victor annonce le montant \`a Charles, il lui transmet 5 informations :
\begin{itemize}
\item V.HRid
\item V.rib
\item le montant
\item un num\'ero de transaction unique pour le vendeur
\item le ``certificat'' de la banque
\end{itemize}
Le message de Victor vers Charles aura donc la forme :\\\textbf{C(V.HRid+V.rib+montant+transactionID, V.Kpriv)+C(V.HRid+V.RIB+V.Kpub, bq.Kpriv)}


Pour payer Victor, Charles doit lui transmettre :
\begin{itemize}
\item V.rib
\item le montant
\item le num\'ero de transaction
\item C(C.HRid+C.rib+c.Kpub, Bq.Kpriv)
\end{itemize}
Donc, le ch\`eque aura la forme : \\\textbf{C(V.RIB+montant+transactionID, C.Kpriv)+C(C.HRid+C.RIB+C.Kpub, bq.Kpriv)}
\subsection{Tests r\'ealis\'es}
Lorsque Charles re\c coit une demande de paiement, il d\'echiffre \\\textbf{C(V.HRid+V.RIB+V.Kpub, Bq.Kpriv)}\\avec Bq.Kpub afin de r\'ecup\'erer V.HRid.
Il v\'erifie que V.HRid est bien celui de Victor (test humain).
S'il pense que c'est bon, il \'emet un ch\`eque \`a Victor avec les autres donn\'ees extraites.


Lorsque Victor re\c coit un ch\`eque, il d\'echiffre \\\textbf{C(C.HRid+C.RIB+C.Kpub, Bq.Kpriv)}\\avec Bq.Kpub afin de r\'ecup\'erer C.RIB et C.Kpub.
Il utilise Kpub pour d\'echiffrer \\\textbf{C(V.RIB+montant+transactionID, C.Kpriv)}\\puis compare V.RIB, montant et transactionID avec celui qu'il a \'emit \`a Charles.
Si les informations sont coh\'erentes, la ch\`eque est valide.


Lorsque la banque re\c coit un ch\`eque, il d\'echiffre \\\textbf{C(C.RHid+C.RIB+C.Kpub, Bq.Kpriv)}\\avec Bq.Kpub afin de r\'ecup\'erer C.RIB et C.Kpub.
Il utilise Kpub pour d\'echiffrer \\\textbf{C(V.RIB+montant+transactionID, C.Kpriv)}\\puis v\'erifie si le couple (V.RIB, transactionID) n'a pas d\'ej\`a \'et\'e utilis\'e dans un autre ch\`eque.
S'il n'y a pas de probl\`emes, il peut d\'ebiter \textit{montant} du compte C.RIB et cr\'editer cette somme sur le compte V.RIB.

\section{Scenarii d'attaque par\'ees}

\subsection{Boris veut intercepter le ch\`eque de Charles}
Borris ne peut pas remplacer C(V.RIB+montant+transactionID, C.Kpriv) pour cr\'editer son propre compte, \'etant donn\'e qu'il ne connais pas C.Kpriv.
La banque refusera le ch\`eque, car la partie remplac\'ee n'a pas \'et\'e chiffr\'e avec C.Kpriv mais avec B.Kpriv.

\subsection{Boris intercepte la facture de Victor et y remplace le RIB par le sien}
Si Boris veut intercepter la facture \'emise par Victor et y mettre son RIB \`a la place de celui de Victor, il devra forger une toute nouvelle facture.
Si Charles ne veut pas se faire avoir, il devra v\'erifier qui est l'\'emeteur de la facture.

\subsection{Boris veut payer Victor avec le ``ch\`equier'' de Charles}
Victor recevra un message qui ressemblera a \\\textbf{C(V.RIB+montant+transactionID, B.Kpriv)+C(C.HRid+C.RIB+C.Kpub, bq.Kpriv)}.\\
Victor ne sauras pas d\'echiffrer C(V.RIB+montant+transactionID, B.Kpriv), \'etant donn\'e qu'il n'a pas \'et\'e chiffr\'e avec C.Kpriv mais avec B.Kpriv.

\subsection{Victor veur encaisser plusieurs fois le ch\`eque de Charles}
Lorsque Victor \'emet une facture, cette facture contient un num\'ero de transaction.
Lorsque Charles \'emet un ch\`eque, ce ch\`eque contient ce m\^eme num\'ero de transaction.
Si Victor essaie d'encaisser plusieurs fois le m\^eme ch\`eque, Banque verra passer plusieurs fois le m\^eme num\'ero de transaction pour le m\^eme RIB.
La banque rejetera donc les r\'e-encaissements d'un m\^eme ch\`eque.

\subsection{Victor veut augmenter le montant du ch\`eque}
Lors de la cr\'eation du ch\`eque, Charles chiffre les informations concernant le destinataire (et notemment le montant) avec sa cl\'e priv\'ee.
Sachant que la cl\'e publique correspondante est chiffr\'ee par la banque dans le certificat, si Victor veut alt\'erer le montant, il faudra qu'il re-chiffre cette partie du ch\`eque avec la cl\'e priv\'ee correspondante (celle de Charles).
Information que seul Charles d\'etiens (en principe).

\subsection{Charles veut r\'eutilser un ch\`eque d\'ej\`a utilis\'e}
Lors de l'\'emission de la facture, Victor g\'en\`ere un num\'ero de transaction unique.
Si Charles veut utiliser un ch\`eque d\'ej\`a utilis\'e, il devra fatalement y apposer un num\'ero de transaction qui a servi pour une autre facture.
Or, si le num\'ero de transaction du ch\`eque ne correspond pas avec celui de la facture, le paiement sera rejet\'e directement par Victor.

\subsection{Charles paye pour une facture qui ne le concerne pas}
H\'elas, dans ce cas, Charles aurait d\^u v\'erifier ce pour quoi il paie.

\end{document}
